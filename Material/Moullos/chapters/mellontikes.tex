Στην παρούσα εργασία, αναπτύξαμε μία φορητή μεθοδολογία για την πρόβλεψη της επικοινωνίας MPI εφαρμογών σε κοινό χώρο διευθύνσεων για point-to-point αλλά και συλλογική επικοινωνία. Βασιζόμενοι σε τεχνικές μηχανικής μάθησης για την δημιουργία μοντέλων επικοινωνίας, ορίσαμε ένα σύνολο χαρακτηριστικών για κάθε τύπο επικοινωνίας που εξετάσαμε. Η μεθοδολογία μας είναι ιδανική για την επιλογή του βέλτιστου configuration πριν την εκτέλεση κάποιας εφαρμογής, αφού μπορεί να προβλέψει τους χρόνους επικοινωνίας χωρίς να απαιτεί πληροφορία που γίνεται διαθέσιμη στο χρόνο εκτέλεσης. Με τη βοήθεια benchmarks εξάγαμε τα σύνολα εκπαίδευσης των μοντέλων, και χρησιμοποιώντας αλγόριθμους επιτηρούμενης μάθησης δημιουργήσαμε σωρεία μοντέλων επικοινωνίας. Αξιολογήσαμε αναλυτικά την ακρίβεια των μοντέλων με τη βοήθεια δύο σύγχρονων MPI εφαρμογών, Jacobi και LULESH, για διάφορα σύνολα εκπαίδευσης αλλά και συνδυασμούς τους.

\paragraph{}
Αναφορικά με την επικοινωνία σημείο προς σημείο, πετύχαμε αξιοσημείωτη ακρίβεια για τον συνδυασμό δύο benchmarks. Το πρώτο, p2p\_SRR, μπορούσε να εκτελέσει μόνο συμμετρική επικοινωνία, όπου κάθε διεργασία που λαμβάνει ένα μήνυμα πρέπει να στείλει πίσω ένα αντίστοιχο μήνυμα. Επιπρόσθετα όλα τα μήκη των μηνυμάτων που ανταλλάσσονται είναι ίσου μεγέθους. Εντούτοις, μας έδινε τη δυνατότητα να διατρέξουμε πολλές παραμέτρους εισόδου για την εξαγωγή ενός μεγάλου σύνολου εκπαίδευσης. Αντίθετα, το δεύτερο, p2p\_AS, οδηγούσε σε ασύμμετρη επικοινωνία με τα μηνύματα να διαφέρουν σε μήκος ανάλογα με τη δομή του πίνακα εισόδου. Παρόλα αυτά, ο συνολικός αριθμός των σημείων που παράγονταν από το benchmark, ήταν ελάχιστος σε σχέση με το p2p\_SRR. Χρησιμοποιήσαμε το συνδυασμό των δύο benchmarks, με σκοπό να εξαλείψουμε τις αδυναμίες τους. Με μεθόδους επιλογής χαρακτηριστικών, αποφανθήκαμε για το πια χαρακτηριστικά έχουν την μεγαλύτερη συσχέτιση με τον χρόνο επικοινωνίας και απορρίψαμε τα υπόλοιπα χαρακτηριστικά από το κοινό σύνολο εκπαίδευσης πετυχαίνοντας ακριβέστερες προβλέψεις. 
\paragraph{}
Επιπρόσθετα, αναλύσαμε εξονυχιστικά τους παράγοντες που δυσχεραίνουν την πρόβλεψη του χρόνου επικοινωνίας για τη συλλογική επικοινωνία ενώ παράλληλα πετύχαμε σημαντικά αποτελέσματα για την πρόβλεψη ενός allreduce που εκτελεί επαναληπτικά η εφαρμογή LULESH. Ωστόσο οι mapping-unaware αλγόριθμοι υλοποίησης και φαινόμενα σχετικά με τη μνήμη, εισάγουν τεράστια δυσκολία στη μοντελοποίηση collective επικοινωνίας. Σε μία μελλοντική επέκταση της μεθοδολογίας μας, θα μπορούσαμε να ασχοληθούμε με collectives τα οποία ανάλογα με την απεικόνιση των MPI διεργασιών στους πόρους των μηχανημάτων προσπαθούν να αξιοποιήσουν χαρακτηριστικά της αρχιτεκτονικής.

\paragraph{}
Μελλοντικά, θα δοκιμάσουμε τη μεθοδολογία μας σε νέες εφαρμογές και μηχανήματα και αρχιτεκτονικές με σκοπό να εξακριβώσουμε πόσο εύκολα μεταφέρεται και ενσωματώνει νέες εφαρμογές. Συγκεκριμένα, σκοπεύουμε να επεκτείνουμε τη μεθοδολογία μας coprocessors, όπως ο Xeon Phi, σε εφαρμογές όπου εκτελούνται αποκλειστικά σε coprocessors ή σε ταυτόχρονη εκτέλεση με κάποιον host επεξεργαστή. 

